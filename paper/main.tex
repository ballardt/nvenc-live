\documentclass[sigconf]{acmart}

\usepackage{booktabs} % For formal tables
\usepackage{multirow}
\usepackage{ctable}
\usepackage{makecell}

% Copyright
%\setcopyright{none}
%\setcopyright{acmcopyright}
%\setcopyright{acmlicensed}
\setcopyright{rightsretained}
%\setcopyright{usgov}
%\setcopyright{usgovmixed}
%\setcopyright{cagov}
%\setcopyright{cagovmixed}


% DOI
%\acmDOI{10.475/123_4}

% ISBN
%\acmISBN{123-4567-24-567/08/06}

%Conference
\acmConference[MMSys'19]{ACM Multimedia Systems}{July 2019}{Amherst, Massachusetts USA}
\acmYear{2019}
\copyrightyear{2019}


%\acmArticle{4}
%\acmPrice{15.00}

% These commands are optional
%\acmBooktitle{Transactions of the ACM Woodstock conference}
%\editor{Jennifer B. Sartor}
%\editor{Theo D'Hondt}
%\editor{Wolfgang De Meuter}


\begin{document}
\title{RATS: Adaptive 360-degree Live Streaming}
%\titlenote{Produces the permission block, and
%  copyright information}
%\subtitle{Extended Abstract}
%\subtitlenote{The full version of the author's guide is available as
%  \texttt{acmart.pdf} document}


\author{Trevor Ballard}
\affiliation{%
  \institution{University of Central Florida}
  \city{Orlando}
  \country{USA}
}
\email{ballardt@knights.ucf.edu}

\author{Carsten Griwodz}
\affiliation{%
  \institution{Simula Research Laboratory University of Oslo}
  \city{Oslo}
  \state{Norway}
}
\email{griff@simula.no}

\author{Amr Rizk}
\affiliation{%
  \institution{Multimedia Communications Lab Technische Universit{\"a}t Darmstadt}
  \city{Darmstadt}
  \country{Germany}}
\email{amr.rizk@kom.tu-darmstadt.de}
%
%% The default list of authors is too long for headers.
%\renewcommand{\shortauthors}{B. Trovato et al.}


\begin{abstract}
Recent works on tiled $360\,^{\circ}$ adaptive bitrate video streaming show significant resource savings at no stalling risk when only parts of the video, e.g., the current and predicted viewport, are transferred in high quality while the rest of the video tiles are transferred in a lower quality. While this is currently feasible for video on demand scenarios, it poses a difficult problem for $360\,^{\circ}$ live streaming as naive methods produce a considerable overhead owing to the lack of tiling support in existing hardware encoders.

In this demo we show Real-time Adaptive Three-sixty Streaming, or RATS, where we utilize GPU-based HEVC encoding to tile, encode, and stitch $360\,^{\circ}$ video at different qualities in real-time. We show measurement results for the encoding speed, resulting encoded output data amount, and output quality for different tiling configurations. While we observe an increase in both encoding time and output file size with the number of desired tile columns, we also see that real-time encoding is ensured for all considered tiling configurations.
\end{abstract}

%
% The code below should be generated by the tool at
% http://dl.acm.org/ccs.cfm
% Please copy and paste the code instead of the example below.
%
%\begin{CCSXML}
%<ccs2012>
% <concept>
%  <concept_id>10010520.10010553.10010562</concept_id>
%  <concept_desc>Computer systems organization~Embedded systems</concept_desc>
%  <concept_significance>500</concept_significance>
% </concept>
% <concept>
%  <concept_id>10010520.10010575.10010755</concept_id>
%  <concept_desc>Computer systems organization~Redundancy</concept_desc>
%  <concept_significance>300</concept_significance>
% </concept>
% <concept>
%  <concept_id>10010520.10010553.10010554</concept_id>
%  <concept_desc>Computer systems organization~Robotics</concept_desc>
%  <concept_significance>100</concept_significance>
% </concept>
% <concept>
%  <concept_id>10003033.10003083.10003095</concept_id>
%  <concept_desc>Networks~Network reliability</concept_desc>
%  <concept_significance>100</concept_significance>
% </concept>
%</ccs2012>
%\end{CCSXML}
%
%\ccsdesc[500]{Computer systems organization~Embedded systems}
%\ccsdesc[300]{Computer systems organization~Redundancy}
%\ccsdesc{Computer systems organization~Robotics}
%\ccsdesc[100]{Networks~Network reliability}
%
%
%\keywords{ACM proceedings, \LaTeX, text tagging}


\maketitle

\section{Introduction}

\section{Details}

\section{Conclusion}

\bibliographystyle{ACM-Reference-Format}
\bibliography{bibliography}

\end{document}
